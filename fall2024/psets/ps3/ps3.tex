\documentclass[11pt]{article}
\usepackage{classTools}

\begin{document}

\psHeader{3}{Wed Oct. 2, 2024 (11:59pm)}

\textbf{Your name: }

\textbf{Collaborators: }

\textbf{No. of late days used on previous psets: }

\textbf{No. of late days used after including this pset: }


The purpose of this problem set is to solidify your understanding of the RAM model (and variants), and the relations between the RAM model, the Word-RAM model, Python programs, and variants. In particular, you will build skills in simulating one computational model by another and in evaluating the runtime of the simulations (both in theory and in practice).

\textit{Note: We STRONGLY recommend typing this problem set (in LaTeX, if possible) -- question 3 will probably be the longest proof we've had you write in the course so far, and unless you have typewriter handwriting, it is much easier for us to grade typed submissions. If we can't read your handwriting, chances are it will lose points.}

\begin{enumerate}
 
    \item (Simulation in practice: RAMs on Python)  
    In the Github repository, we have given you a partially written Python implementation of a universal RAM Model simulator.  Your task is to fill in the missing parts of the code to obtain a complete universal RAM simulator.
     Your simulator should take as input a RAM Program $P$ and an input $x$, and simulate the execution of $P$ on $x$, and return whatever output $P$ produces (if it halts).  The RAM Program $P$ is given as a Python list $[v,C_0,C_1,\ldots,C_{\ell-1}]$, where $v$ is the number of variables used by $P$.  For simplicity, we assume that the variables are named $0,\ldots,v-1$ (rather than having names like ``tmpptr'' and ``insertvalue''), but you can introduce constants to give names to the variables.  The $0$\textsuperscript{th} variable will always be $\inputlen$, the $1$\textsuperscript{st} variable $\outputpointer$, and the $2$\textsuperscript{nd} variable $\outputlen$.  A command $C$ is given in the form of a list of the form $[\cmd]$, $[\cmd,i]$, $[\cmd,i,j]$, or $[\cmd,i,j,k]$, where $\cmd$ is the name of the command and $i,j,k$ are the indices of the variables or line numbers used in the command.  For example,  the command $\var_i = M[\var_j]$ would be represented as $(\READ,i,j)$.  See the Github repository for the precise syntax as well as some RAM programs you can use to test your simulator.

    \item (Empirically evaluating simulation runtimes and explaining them theoretically)  

Consider the following two RAM programs:

\begin{algorithm}[H]
\Input{A single natural number $N$ (as an array of length 1)}
\Output{$13^{2^N+1}$ (as an array of length 1)}
\Variables{$\inputlen, \outputpointer, \outputlen, \counter, \result$}
\setcounter{AlgoLine}{-1}
$\zero = 0$\;
$\one = 1$\;
$\thirteen = 13$\;
$\outputlen = 1$\;
$\outputpointer = 0$\;
$\result = 13$\;
$\counter = M[\zero]$\;
\Indp
 IF $\counter == 0$ GOTO \ref{line:done}\; \label{line:loop}
$\result = \result \times \result$\;
$\counter = \counter - \one$\;
IF $\zero == 0$ GOTO \ref{line:loop}\;
\Indm
$\result = \result \times $\thirteen\; \label{line:done}
$M[\outputpointer]=\result$\;
\end{algorithm}

\begin{algorithm}[H]
\Input{A single natural number $N$ (as an array of length 1)}
\Output{$13^{2^N+1} \bmod 2^{32}$ (as an array of length 1)}
\Variables{$\inputlen, \outputpointer, \outputlen, \counter, \result, \temp, \W$}
\setcounter{AlgoLine}{-1}
$\zero = 0$\;
$\one = 1$\;
$\thirteen = 13$\;
$\outputlen = 1$\;
$\outputpointer = 0$\;
$\result = 13$\;
$\W = 2^{32}$\;
$\counter = M[\zero]$\;
\Indp
IF $\counter == 0$ GOTO \ref{line:done2}\; \label{line:loop2}
$\result = \result \times \result$\;
$\temp = \result / \W$\;
$\temp = \temp \times \W$\;
$\result = \result - \temp$\;
$\counter = \counter - \one$\;
IF $\zero == 0$ GOTO \ref{line:loop2}\;
\Indm
$\result = \result \times \thirteen$\;
\label{line:done2}
$\temp = \result / \W$\;
$\temp = \temp \times \W$\;
$\result = \result - \temp$\;
$M[\outputpointer]=\result$\; 
\end{algorithm}

\begin{enumerate}
    \item Exactly calculate (without asymptotic notation) the RAM-model running times of the above algorithms as a function of $N$.
    Which one is faster? \label{itm:RAMtime}    
    \item Using your RAM Simulator, run both RAM programs on inputs $N=0,1,2,\ldots,15$ and graph the actual running times (in clock time, not RAM steps).  (We have provided you with some timing and graphing code in the Github repository.) Which one is faster?  \label{itm:realtime}  
    
    \item Explain the discrepancies you see between Parts~\ref{itm:RAMtime} and \ref{itm:realtime}.  (Hint: What do we know about the relationship between the RAM and Word-RAM models, and why is it relevant to how efficiently the Python simulation runs?) 
    
    \item (optional\footnote{This problem won't make a difference between N, L, R-, and R grades.}) Give a theoretical explanation of the shapes of the runtime curves you see in Part~\ref{itm:realtime}, by providing explicit formulas for the asymptotic runtimes of the two programs (in clock time). You may need to do some research online and/or make guesses about how Python operations are implemented to come up with your estimates. 

\end{enumerate}

\item (Simulating Word-RAM by RAM) For every Word-RAM program $P$, there is a RAM program $P'$ that simulates $P$ in the sense that:
\begin{enumerate}
    \item $P'$ halts on $(w,x)$ iff $P[w]$ halts on $x$, and 
    \item If $P[w]$ crashes, then $P'$ halts with $\outputpointer=\outputlen=0$. (We are using this output setting to indicate \crash, since the RAM model does not have any crashing in its semantics.)
    \item If $P[w](x)$ halts without crashing, then the output of $P'(x,w)$ equals the output of $P[w](x)$.
     Furthermore,   
       $$\Time_{P'}(x) = O\left(\Time_{P[w]}(x)+n+w\right),$$
where $n$ is the length of $x$.

(This was stated without proof in Lecture Notes 8.) 

\end{enumerate}

Your proof should use an {\em implementation-level} description, similar to the proof that RAM programs can be simulated by ones with at most $c$ registers in Lecture 7.  Recall that Word-RAM programs have a finite but changing memory size $S$ and a read-only variable $\wordlen$; you may want to start your simulation by calculating $S$ and $2^{\wordlen}$.  Then think about how each operation of a Word-RAM program $P$ can be simulated in a RAM program $P'$, taking care of any differences between their semantics in the Word-RAM model vs. the RAM model. Don't forget MALLOC!

\item (reflection) Discuss the value (or lack thereof) that you think computational models, and the RAM and Word-RAM models in particular, have for computer science.  All opinions are valid, as long as they show serious thought and are backed by specific justifications.

\textit{Note: As with the previous psets, you may include your answer in your PDF submission, but the answer should ultimately go into a separate Gradescope submission form.}

\item Once you're done with this problem set, please fill out \href{https://forms.gle/TG8cG4LWN3S4T6za6}{this survey} so that we can gather students' thoughts on the problem set, and the class in general. It's not required, but we really appreciate all responses!

\end{enumerate}


\end{document}
